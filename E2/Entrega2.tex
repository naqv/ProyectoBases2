\documentclass[a4paper]{article}

\usepackage[english]{babel}
\usepackage[utf8x]{inputenc}
\usepackage{amsmath}
\usepackage{graphicx}
\usepackage[colorinlistoftodos]{todonotes}
\begin{document}
\title{\LARGE{Universidad Simón Bolivar} \hspace{5 cm}
\LARGE{Paradigmas en Modelaje de Bases de Datos}}

\begin{figure}
\centering
\includegraphics[width=0.5\textwidth]{probar3.jpg}
\end{figure}

  \vspace{3 cm}
\author{ \Huge{UNIVERSO DISCURSO}\vspace{1 cm} \\ Karen Piotrowski \\ Natacha Quintero \\ Yeiker Vazquez}


\maketitle {}


\vspace{15 cm}



\section{ \underline{Introducción:} }

   \setlength{\parindent}{0,5cm}
\setlength{\parskip}{\baselineskip} 

    \hspace{0,4 cm} Cada año, millones de personas viajan a distintas partes del mundo para hacer turismo y así recorrer nuevos horizontes. Ir a un lugar desconocido es desafiante si no se posee información sobre la ciudad deseada. Con la herramienta  \textit{YourCity} se busca facilitar al turista la visita a una ciudad desconocida, viajando a través de las rutas fijas que ofrece la aplicación u ofreciéndole la oportunidad de diseñar su propia ruta dinámica. En cada ruta, el turista puede hacer paradas en hitos utilizando las vías que acceden al mismo con el transporte que se ofrezca.

	Un turista desea conocer una ciudad nueva y evitar experiencias desagradables, como no encontrar entrada en un hito, perderse en la ciudad o quedarse sin efectivo; por ello \textit{YourCity} ofrece paquetes que incluyen las entradas a los hitos pertenecientes a una ruta fija así como beneficios como el transporte, comida, visita guiada, entre otros.

En un enfoque técnico, para modelar la base de datos se utiliza el diagrama OMT (Object Modeling Technique) que utiliza metodologías de análisis y diseño orientadas a objetos; dichas características proporcionan más flexibilidad y facilidad de uso lo cual es esencial para que la base de datos perdure en el tiempo. En dicho diagrama se muestran clases que representan a los entes en el universo de discurso, las mismas se asocian entre ellas para reflejar su comportamiento.

\setlength{\parindent}{0,5cm}
\setlength{\parskip}{\baselineskip} 
\vspace{10 cm}





\section{\underline{Descripción de las clases::}}
 \setlength{\parindent}{0,5cm}
\setlength{\parskip}{\baselineskip} 



\begin{itemize}

\item{
	\textbf{\textit{Hito}}: Es una atracción turística ubicada en un punto geográfico dentro de una ciudad que puede ser visitado por el público general. 
  		
     \begin{itemize}   
  		\item[$-$] Lista de atributos:

	\begin{itemize}
		\item[$o$] Nombre: Atributo que identifica al Hito.
        \item[$o$] Longitud.
        \item[$o$] Latitud.
        \item[$o$] Altitud.
        \item[$o$] Descripción: información breve sobre un hito.
        \item[$o$] Estado: Atributo que indica si un hito está en funcionamiento o no. El dominio de estado es: disponible, en reparación y clausurado temporalmente.
        \item[$o$] Categoría[0..*]: En este atributo multivaluado se especifica el conjunto de géneros del Hito. Su dominio es: entretenimiento, deporte, cultura, infantil, ciencia, historia, música.
        \item[$o$] Edad apropiada: Edad mínima para poder acceder al hito.
\item[$o$] Página Web:  En caso de existir, este atributo es un link a la página web del hito.
\item[$o$] Teléfono[0..*]: teléfonos de contacto del hito.
\item[$o$] Email: correo electrónico asociado al hito.
\item[$o$] CostoMonedaLocal: precio de la entrada en la moneda del país. El precio puede ser cero.
\item[$o$] Horarios: Atributo de texto que incluye la información del horario de apertura del hito para cada día. Además incluye información de los días que no trabajan o de aquellos días que tienen un horario diferente.
\item[$o$] Foto: Contiene la dirección completa del archivo en el que se encuentra la foto en la computadora que funciona como servidor de la aplicación. Es de tipo PATH.

	\end{itemize}


	\item[$-$] Lista de operaciones

	\begin{itemize}
		\item[$o$] calcularCostoEnUSD(relacionLocalUSD: Numeric): se encarga de calcular el costo del ticket de un hito en dólares americanos; para ello necesita la relación entre la moneda local y el dólar americano.
	
	\end{itemize}



    \end{itemize}

}


\item{
	\textbf{\textit{Vía}}: Es un camino que conduce de un punto a otro en la ciudad, se utiliza para acceder a los hitos 
  		
     \begin{itemize}   
  		\item[$-$] Lista de atributos:

	\begin{itemize}

        \item[$o$] Nombre: Atributo que identifica a la vía.
\item[$o$] Tipo: Este atributo corresponde a la clasificación de la vía, es decir, su dominio puede ser: calle, avenida, caminería, autopista, elevado, camino, puente.
\item[$o$] TiposTransporte:   Este es un atributo multivaluado en el cual se tienen los distintos tipos de vehículos que podrían transitar dicha vía. El dominio del transporte es: a pie, automóvil, autobús, taxi, bicicleta, metro y tren. 
\item[$o$] InicioLatitud: Latitud del punto de inicio de una vía.
\item[$o$] InicioLongitud: Longitud del punto de inicio de una vía.
\item[$o$] finLatitud: Latitud del punto final de una vía.
\item[$o$] finLongitud: Longitud del punto final de una vía


	\end{itemize}


    \end{itemize}

}


\item{
	\textbf{\textit{Ruta}}: Es un recorrido entre dos puntos de una ciudad que incluye hitos y permite recorrer un conjunto de hitos de una manera específica. 
  		
     \begin{itemize}   
  		\item[$-$] Lista de atributos:

	\begin{itemize}
		 
        
      \item[$o$] Nombre: Atributo que identifica a una ruta.
\item[$o$]Descripción: Breve descripción opcional que describe características de la ruta en cuestión.
\item[$o$]Latitud del punto de inicio: atributo numérico que indica la latitud del punto inicial de una ruta.
\item[$o$]Longitud del punto de inicio: atributo numérico que indica la longitud del punto inicial de una ruta.
\item[$o$]Latitud del punto final: atributo numérico que indica la latitud del punto final de una ruta.
\item[$o$]Longitud del punto final: atributo numérico que indica la longitud del punto final de una ruta.
\item[$o$]Número de visitas: representa el número de personas que han realizado la ruta.
Fecha de Ingreso: Atributo de fecha que indica cuándo fue creada una ruta en la base de datos.



	\end{itemize}


	\item[$-$] Lista de operaciones

	\begin{itemize}
		\item[$o$]  calcularAntiguedad(): Calcular la cantidad de tiempo que ha transcurrido desde la fecha de ingreso de la ruta en la base de datos hasta la fecha actual.
\item[$o$]hallarDuracionAproximada(): Mediante esta operación se puede hallar la duración aproximada de recorrido de la ruta calculando un promedio que se obtiene de los atributos “horaInicio” y “horaFin” de la asociación “toma” entre usuario y ruta.

	
	\end{itemize}



    \end{itemize}

}



\item{
	\textbf{\textit{Valoración}}: Es una representación de la opinión de un usuario sobre un hito, ruta o ciudad. El usuario debe colocar de forma obligatoria una puntuación numérica del 1 al 5
  		
     \begin{itemize}   
  		\item[$-$] Lista de atributos:

	\begin{itemize}
		 
        
         \item[$o$]Puntuación: Opinión expresada numéricamente con respecto a la apreciación  general del hito, ruta o ciudad visitada por el usuario.
\item[$o$]Seguridad: Opinión expresada numéricamente con respecto a la apreciación de la seguridad en un hito, ruta o ciudad visitada por el usuario.
\item[$o$]Comentario: Espacio de texto para especificar la experiencia del usuario en el hito.
\item[$o$]Relación Precio-valor: Opinión expresada numéricamente con respecto a la relación entre el precio y el valor (calidad de experiencia) que puede tener un hito, ruta o ciudad visitada por el usuario.



	\end{itemize}


    \end{itemize}

}



\item{
	\textbf{\textit{Usuario}}:Un usuario es la persona a quien se dirige el portal, la cual puede utilizarlo para consultar la información acerca de rutas turísticas urbanas  
  		
     \begin{itemize}   
  		\item[$-$] Lista de atributos:

	\begin{itemize}
		
        \item[$o$] Alias: Es el identificador del usuario dentro del portal el cual debe ser único.
\item[$o$] Nombres: Es un String que contiene los nombres completos del usuario.
\item[$o$] Apellidos: Es un String que contiene los apellidos completos del usuario.
\item[$o$] Email: Es la dirección de correo electrónico con la que el usuario se registra al portal y debe ser única.
\item[$o$] Género: Indica el sexo del usuario.
\item[$o$] Contraseña: Es un String necesario para hacer login en el portal. La contraseña y el nombre de usuario deben estar relacionados para poder ingresar al sistema.
\item[$o$] EsEstudiante: Un booleano que representa si el usuario es estudiante.
\item[$o$] Fecha de Nacimiento: Se especifica el día, mes y año de nacimiento del usuario.
\item[$o$] Biografía: Es un espacio de texto que el usuario puede rellenar a su gusto.
\item[$o$] Interés: Es un atributo multivaluado que contiene los tipos de hitos que le interesan al usuario. El dominio de un interés puede ser: entretenimiento, deporte, cultura, infantil, ciencia, historia.




	\end{itemize}


	\item[$-$] Lista de operaciones

	\begin{itemize}
		
        
        
       \item[$o$] calcularEdad(): toma la fecha actual y la resta con la fecha de nacimiento del usuario y retorna la edad.
\item[$o$] esTerceraEdad(): retorna si el usuario ya ha llegado a la tercera edad

        
        
	\end{itemize}
    
    
    \item[$-$] Restricciones explícitas:    


      \begin{itemize}

     \item[$o$]  La fecha actual menos la fecha de nacimiento no puede ser menor de 15 años. Es decir, que un usuario debe tener por lo mínimo 16 años para ser un usuario.

      \end{itemize}

    \end{itemize}

}



\item{
	\textbf{\textit{Ciudad}}:  Una ciudad es una entidad urbana, definida en un área geográfica
  		
     \begin{itemize}   
  		\item[$-$] Lista de atributos:

	\begin{itemize}
		\item[$o$] Nombre: Atributo que sirve como identificador principal.
\item[$o$]País: Nombre del país al que pertenece una la ciudad.
\item[$o$]NombreMonedaLocal: Este atributo contiene el nombre de la moneda local.
\item[$o$]ValorMonedaLocal: Este atributo tiene el valor de la moneda local en relación al dólar.

        



	\end{itemize}



    \end{itemize}

}








\item{
	\textbf{\textit{Servicio}}: Un servicio es un conjunto de lugares o actividades destinados a incrementar la satisfacción de un usuario al momento de visitar un hito o tomar una ruta por determinadas vías.
 
  		
     \begin{itemize}   
  		\item[$-$] Lista de atributos:
        
        \begin{itemize}
       \item[$o$]Nombre: Identifica al servicio, sin embargo, pueden haber varios servicios con un mismo nombre como es el caso de un Mc Donalds.
\item[$o$]Categoría [0..*]: Con este atributo se puede especificar la finalidad de dicho servicio. El dominio en este caso es el siguiente: salud, entretenimiento, comida, transporte, estación de servicio, entre otros.
\item[$o$]Número de contacto [0..*]: En caso de tener, un número telefónico que permita contactar a dicho servicio.
\item[$o$]Email: En caso de tenerlo, correo electrónico de dicho servicio.
\item[$o$]PáginaWeb: En caso de tenerlo, página web de dicho servicio.
\item[$o$]Dirección: dirección del servicio (puede ser no específica).
\item[$o$]EsGratuito: Es un atributo booleano que indica si el servicio es gratuito o no.
\item[$o$]Descripción: Un campo de texto en el cual se explica brevemente las características de dicho servicio. Como en el caso de un restaurante, mediante este atributo puede explicar brevemente el tipo de comidas que ofrece y el público hacia el cual está dirigido.
        \end{itemize}
      
   


    \end{itemize}

}





\item{
	\textbf{\textit{Paquete}}: Son servicios (mayormente pagos) ofrecidos en el portal para dar la facilidad al usuario de obtener, mediante una sola compra, entradas en los distintos hitos a visitar, servicio de transporte, comida y guia turistico, entre otros servicios disponibles.
  		
     \begin{itemize}   
  		\item[$-$] Lista de atributos:
        
        
        
        
        \begin{itemize}
       \item[$o$]Nombre: Atributo que identifica un nombre de un paquete. El nombre de un paquete va a ser creativo para atraer la atención de un usuario
\item[$o$]Precio: Es el monto que paga un usuario estándar. Un usuario estándar es aquel que no es estudiante, no pertenece a la tercera edad y no es un niño. \item[$o$] CostoEstudiante: Es precio del paquete que paga un usuario o un acompañante que sea estudiante
\item[$o$] CostoTerceraEdad: Es el precio que paga un usuario o acompañante que pertenezca a la tercera edad.
\item[$o$] CostoNino: Es el precio que paga un acompañante al ser un niño.
\item[$o$] Descripción: En este atributo se explica brevemente las características que el paquete incluye.

        \end{itemize}
      
      \item[$-$] Lista de operaciones
   
   
      \begin{itemize}
      \item[$o$] calcularTarifaMinima(): Esta operación permite calcular el monto básico del paquete para una persona estándar haciendo uso de los precios en dólares de los hitos que se encuentran en este paquete
   \end{itemize}


    \end{itemize}

}


\item{
	\textbf{\textit{Evento}}: Sucesos que se llevan a cabo en un hito determinado.
  		
     \begin{itemize}   
  		\item[$-$] Lista de atributos:
        
        
        
        
        \begin{itemize}
       \item[$o$]Nombre: Atributo que identifica al evento.
 \item[$o$]Categoría: Este atributo clasifica al evento dentro de un dominio específico.Dicho dominio es el siguiente: musical, entretenimiento, cultural, histórico, social,festivo.
 \item[$o$]CostoMonedaLocal: Indica el precio de dicho evento en la moneda local
 \item[$o$]Descripción: atributo que explica brevemente las características de un evento

        \end{itemize}
      
      \item[$-$] Lista de operaciones
   
   
      \begin{itemize}
      \item[$o$] calcularCostoEnUSD(relacionLocalUSD): se encarga de calcular el costo de entrada del evento en dólares americanos; para ello necesita la relación entre la moneda local y el dólar americano.
   \end{itemize}


    \end{itemize}

}

\end{itemize}


\section{\underline{Descripción de asociaciones:} }



\begin{itemize}

\item{

	\textbf{\textit{Contiene (Hito - contiene - Hito):}}Asociación recursiva que indica la contención de un hito dentro de otro hito.
  		
        
        
 \begin{itemize}
        
  \item Roles: En este caso se poseen dos roles: Externo e interno. El primero corresponde a un Hito padre que contiene otros Hitos internos.
\item Cardinalidades: Para cada Hito con rol externo se pueden tener 0..* Hitos internos. Y para cada Hito interno se puede tener 1 Hito externo, es decir, puede tener solo un padre.
\item Ejemplo de instancia: Contiene = {(USB, BibliotecaUSB), (USB,Cromovegetal)}
\item Restricciones explícitas: Un Hito no se contiene a sí mismo, es decir, los nombres tanto de Hito interno como del externo no deben coincidir.{itemize} 
  	
   \end{itemize}


}


\item{

	\textbf{\textit{Contrata (Usuario - compra - Paquete):}}Esta asociación indica que un usuario puede comprar algún paquete.
  		
        
        
 \begin{itemize}
        
  \item Lista de atributos: Tiene un atributo multivaluado llamado “acompañante” en el cual se especifica el número de acompañantes y el tipo de acompañante. Por ejemplo: 3 niños, 2 normales, 4 tercera edad, 1 estudiante.
  
  \begin{itemize}
  \item[$o$]Acompañante:
\begin{itemize}
 \item Tipo: Atributo que indica qué usuario es, es decir, puede tomar valores
del siguiente dominio: niño, tercera edad, estudiante, normal.
\item Número: Atributo numérico que indica la cantidad de personas por
tipo. 
\end{itemize}

\end{itemize}



\item Lista de operaciones:

\begin{itemize}
\item[$o$] CalcularPrecioTotal(): Esta operación se calcula a partir del número de
acopanantes y su tipo y el precio para cada tipo de acompañante.

\end{itemize}


\item Cardinalidades: Un usuario puede contratar 0..* paquetes, y un paquete puede ser contratado por 0..* usuarios.

\item Ejemplo de instancia: Contrata = {(Luis, “Tour de los museos de arte Parisinos”, {(niño,3), (estudiante,1), (tercera edad,2), (normal,3)}}
  
  
  
   \end{itemize}


}


\item{

	\textbf{\textit{Crea (Usuario - crea - Dinámica):}}Esta asociación indica que un usuario puede crear una ruta dinámica.
  		
        
        
 \begin{itemize}
        
  \item Cardinalidades: Un usuario puede crear entre 0..* rutas dinámicas, y una ruta dinámica es creada únicamente por 1 usuario.
\item Ejemplo de instancia: Crea = {(Luis, “Paseo por mis museos favoritos de Berlín”)}
   \end{itemize}


}


\item{

	\textbf{\textit{Es (Vía - Es - Hito): }}Esta asociación indica si una via es un Hito o no.
  		
        
        
 \begin{itemize}
        
  \item Cardinalidades: Una vía puede ser o no ser un Hito es decir una vía es 0..1 veces un
Hito, y un Hito es 0..1 veces una vía, es decir, un Hito puede ser o no ser una vía.
\item Ejemplo de instancia: Es = {(Puente de San Francisco, Puente de San Francisco)} 
\item Restricciones explícitas: Para que una Vía sea un Hito, el nombre de ambas clases debe coincidir.
   \end{itemize}


}


\item{

	\textbf{\textit{EstaEn (Ruta - estaEn - Ciudad):}}Esta asociación indica si una ruta esta en una ciudad.
  		
        
        
 \begin{itemize}
        
  \item Cardinalidades: Una Ruta puede estar en 1 Ciudad y en una Ciudad pueden estar 1..*
Rutas.
\item  Ejemplo de instancia: EstaEn = {(Ruta de museos de Guerra, Berlin), (Ruta de teatros de la ópera de Berlín, Berlín)}
   \end{itemize}


}




\item{

	\textbf{\textit{Incluye (Paquete - incluye - Fija):}}Esta asociación representa que un Paquete incluye Rutas fijas.
  		
        
        
 \begin{itemize}
        
  \item Cardinalidades: Un Paquete puede incluir 1 Ruta fija, y una Ruta fija puede estar incluidas en 0..* Paquetes.
\item Ejemplo de instancia: Incluye = {(“Super recorrido por plazas con comida y bebida incluida”, “Tour por las plazas de Berlín”)}
   \end{itemize}


}




\item{

	\textbf{\textit{Ocurre (Evento - ocurre - Hito): }}Esta asociación indica que los eventos deben ocurrir en Hitos y que en los Hitos pueden o no haber Eventos.
  		
        
        
 \begin{itemize}
        
  \item Lista de atributos: Esta asociación tiene como atributos la fecha inicio del evento y la fecha fin.
  
  \begin{itemize}
\item[$o$] FechaInicio: Fecha en la que inicia dicho evento.
\item[$o$] FechaFin: Fecha en la que termina el evento. Esta fecha podría no estar especificada ya que podría no conocerse cuando sera el fin de dicho evento.

\end{itemize}
 \item Cardinalidades: Un Evento puede ocurrir únicamente en 1 Hito, sin embargo, en un Hito
pueden ocurrir 0..* Eventos.  

\item Ejemplo de instancia: Ocurre = {(“Festival de verano”, “Plaza Bolívar”)} 
 \item Restricciones explícitas:
 
 \begin{itemize}
\item[$o$] Un evento no puede ocurrir en un Hito cuyo estado no sea “Disponible”
\item[$o$] La fecha inicio no puede ser mayor que la fecha fin.
\end{itemize}
   \end{itemize}


}


\item{

	\textbf{\textit{RutaEn (Ruta - rutaEn - Hito):}} Con esta asociación lo que se busca representar es que dentro de un Hito pueden haber a su vez Rutas ya que un Hito puede tener Hitos y pueden haber diversas maneras de recorrer dichos Hitos internos.
        
        
 \begin{itemize}
        
  \item Cardinalidades: Un Hito puede tener 0..* Rutas dentro de sí, y una Ruta puede estar 0..1 veces en un Hito, esto es, una Ruta puede estar o no estar dentro de un Hito. Si la Ruta está dentro del Hito no puede estar en otros Hitos ya que consideraría como una Ruta interna particular de un Hito.
\item Ejemplo de instancia: RutaEn = {(“Paseo por los jardines de la USB”, USB)}
   \end{itemize}


}





\item{

	\textbf{\textit{SeAccedePor (Hito - SeAccedePor - Via): }}Esta asociación indica que un Hito puede ser accedido mediante ciertas vías.
  		
        
        
 \begin{itemize}
        
  \item Cardinalidades: Un Hito se puede acceder por 1..*Vías, y una vía puede servir de acceso a 0..* Hitos.
 \item Ejemplo de instancia: SeAccedePor = {(Plaza Francia, Avenida Luis Roche)}
   \end{itemize}


}



\item{

	\textbf{\textit{SeEncuentra (Via - seEncuentra - Ciudad):}}
  		Esta asociación indica las vías que se encuentran dentro de una ciudad.
        
        
 \begin{itemize}
        
  \item Cardinalidades: Una ciudad puede tener 1..* Vías, mientras que una Vía puede encontrarse solo en una Ciudad.
  \item Ejemplo de instancia: SeEncuentra = {(“Avenida las Delicias”, “Maracay”)}
   \end{itemize}


}




\item{

	\textbf{\textit{Tiene (Hito - tiene - Servicio): }}Esta asociación sirve para explicar si un Hito tiene o no Servicios asociados a él. Como por ejemplo el servicio de restaurante dentro de un museo
  		
        
        
 \begin{itemize}
        
  \item Cardinalidades: Un Hito puede tener 0..* Servicios asociados a él, mientras que un mismo servicio puede estar asociado también a 0..* Hitos.
\item Ejemplo de instancia: Tiene = {(Museo, Restaurante), (Museo, Baño)}
   \end{itemize}


}





\item{

	\textbf{\textit{TieneServicio (Vía - tiene - Servicio): }}Esta asociación representa aquellos servicios que se encuentran en medio de una vía o se acceden por medio de ella.
  		
        
        
 \begin{itemize}
        
  \item Cardinalidades: Una Vía tiene entre 0..* Servicios y un Servicio puede estar en 0..* Vías.
 \item Ejemplo de instancia: TieneServicio = {(“Avenida las Delicias”, Gasolinera PDV)}
   \end{itemize}


}


\item{

	\textbf{\textit{Toma (Usuario -toma-Ruta):}}Aquí van a haber unos atributos que indiquen hora de inicio y hora de fin del recorrido de la ruta.
  		
        
        
 \begin{itemize}
        
  \item Lista de atributos:
  
  
  \begin{itemize}
\item[$o$]EsAprobada: Atributo que permite al usuario que toma la Ruta, sólo si es una
ruta dinámica, aprobarla para que esta pueda ser promovida como una ruta fija.
\item[$o$]HoraInicio: Atributo que indica la hora en la que el Usuario tomó dicha Ruta.
\item[$o$] HoraFin: Atributo que indica la hora en la que el Usuario finalizó dicha Ruta.

\end{itemize}

\item Cardinalidades: Un Usuario puede tomar entre 0..* Rutas, y una Ruta puede ser tomada por 0..* Usuarios.
\item Ejemplo de instancia: Toma = {(Luis, “Los teatros favoritos de Teresa”, TRUE, 7:00 AM, 8:00
PM)}



\item Restricciones explícitas:
\begin{itemize}
\item[$o$]Un usuario puede aprobar una ruta sólo si esta ruta es dinámica, de lo contrario, el valor por defecto para el atributo “EsAprobada” sera FALSE en caso de que sea una ruta fija y no tendrá efecto alguno.
\item[$o$]La HoraFin debe ser mayor que la HoraInicio.

\end{itemize}
   \end{itemize}
   
}


item{

  
	\textbf{\textit{ValoraCiudad (Ternaria entre: Usuario, valoración, Ciudad):}}
     un usuario puede valorar una ciudad
        
        
 \begin{itemize}
        
  \item Cardinalidad:
  
  
  \begin{itemize}
\item[$o$]Un Usuario con una Ciudad puede realizar 0..* valoraciones, ya que puede no valorarla pero también puede que la visite varias veces y dé una valoración nueva.
\item[$o$] Una Valoración de una Ciudad puede haber sido realizada por 1 solo Usuario.
\item[$o$]Un Usuario que da una Valoración puede corresponder sólo a 1 Ciudad.

\end{itemize}




\item Ejemplo de instancia: ValoraCiudad = {(Luis, 5, 5 ,5, “Es la ciudad más hermosa que he conocido”, Venecia)} (Los valores numéricos corresponden a las valoraciones que el usuario dio sobre la ciudad)

 \end{itemize}



}





item{

  
	\textbf{\textit{ValoraHito (Ternaria entre: Usuario, valoración,Hito)::}}
     un usuario puede valorar un hito 
        
        
 \begin{itemize}
        
  \item Cardinalidad:
  
  
  \begin{itemize}
\item[$o$]Un Usuario con un Hito puede realizar 0..* valoraciones, ya que puede no valorar el Hito pero también puede valorarlo varias veces en distintos momentos (ya que el Hito podría mejorar o empeorar).
\item[$o$]Una Valoración de un Hito puede haber sido realizada por 1 solo Usuario.
\item[$o$] Un Usuario que da una Valoración puede corresponder sólo a 1 Hito.

\end{itemize}




\item Ejemplo de instancia: ValoraHito = {(Luis, 3, 5 ,3, “Esta plaza realmente no es muy segura, pero al menos es gratuita, sin embargo esta muy descuidada”, Plaza Francia)}

 \end{itemize}



}



item{

  
	\textbf{\textit{ValoraRuta (Ternaria entre: Usuario, valoración, Ruta:}}
     un usuario puede valorar una ruta
        
        
 \begin{itemize}
        
  \item Cardinalidad:
  
  
  \begin{itemize}
\item[$o$]Un Usuario con una Ruta puede realizar 0..* valoraciones, ya que puede no valorar la Ruta, sin embargo, puede valorarlo varias veces en distintos momentos (ya que la Ruta podría mejorar o empeorar).
\item[$o$] Una Valoración de una Ruta puede haber sido realizada por 1 solo Usuario.
\item[$o$] Un Usuario que da una Valoración puede corresponder sólo a 1 Ruta

\end{itemize}




\item Ejemplo de instancia: ValoraRuta = {(Luis, 5, 5 ,5, “Es la ciudad más hermosa que he conocido”, Venecia)} (Los valores numéricos corresponden a las valoraciones que el usuario dio sobre la ciudad)

 \end{itemize}



}

\end{itemize}


\section{\underline{Descripción de asociaciones:} }


\begin{itemize}



  
\item{	

\textbf{\textit{Agregación entre Ruta e Hito: }}Esta asociación de agregación indica que una ruta esta compuesta de un conjunto de hitos en un orden en específico. En particular este orden define el recorrido de una ruta. Aunque un hito es parte de una ruta, dicho Hito existe aunque no esté en una Ruta.
        
 \begin{itemize}
 
 \item[$o$]Cardinalidades: Una ruta puede tener entre 1..* hitos, y un hito puede pertenecer a 0..* rutas.
\item[$o$] Ejemplo de instancia: {(“Ruta Amigo Deportista”, “UCV”), ( “Ruta Amigo Deportista”, “Los
Próceres”)}
 
 \end{itemize}
 
 
 
}





  
\item{	

\textbf{\textit{Agregación entre Ruta y via: }}Esta asociación de agregación indica que una ruta esta compuesta de un conjunto de vías.
        
 \begin{itemize}
 
 \item[$o$]Cardinalidades: Una ruta puede tener entre 1..* vías, y una vía puede pertenecer a 0..* rutas.
 \item[$o$] Ejemplo de instancia: {(“Ruta Playera”,“Carretera de la Guaira”), (“Ruta Playera”,“Taguao”)}
 
 
 \end{itemize}
 
 }

\end{itemize}

\section{\underline{Descripción de Generalizaciones:} }




\begin{itemize}


\item{	

\textbf{\textit{Ruta dinámica: }}s una especialización de ruta. Una ruta es dinámica cuando el usuario decide cómo recorrerla y que hitos visitar. Un usuario puede proponerla como fija y de esta forma otros usuarios pueden tomarla y aprobarla. Dependiendo del número de aprobaciones que tenga , puede volverse fija .
        
 \begin{itemize}
 
 \item[$o$] Lista de atributos:
 
 \begin{itemize}
\item[$o$]PropuestaFija : Este atributo indica si la ruta dinámica hecha por un usuario es
propuesta para ser una ruta fija. En el caso de ser propuesta, debe analizarse el
número de aprobaciones que tiene por parte de otros usuarios que la recorran. 
\end{itemize}

\item[$o$] Lista de operaciones:

\begin{itemize}
\item[$o$]calcularAprobaciones(): NUMERIC
 \end{itemize}
 
 
 \end{itemize}

 
 }




\item{	

\textbf{\textit{Paquete: }}Un paquete es una especialización de servicio. Consiste en un combo de promocion de rutas, hitos y vías que además pueden incluir servicios extras como transporte o comida, con precios especiales. Dependiendo si el usuario es un niño, pertenece a la tercera edad o es estándar puede recibir descuentos especiales.
 \begin{itemize}
 
 
 
 \item[$o$] Lista de atributos:
 
 \begin{itemize}
\item[$o$]Nombre: Atributo que indica el nombre del paquete. CostoEstudiante: Precio que paga un estudiante por el paquete.
\item[$o$] CostoEstandar: Precio que paga un usuario que no es ni estudiante, ni nino y tampoco pertenece a la tercera edad.
\item[$o$] CostoTerceraEdad: Precio que paga un usuario que pertenece a la tercera edad 
\item[$o$] CostoNino: precio que paga un acompanante del usuario al ser un niño. \item[$o$] Descripción: Breve descripción del paquete 
\end{itemize}

\item[$o$] Lista de operaciones:

\begin{itemize}
\item[$o$]calcularTarifaMinima(): operación que suma los precios de cada hito que pertenecen
a una ruta escogida por un usuario. Esta suma representa el precio base ( sin promoción ) que paga un usuario estándar.
 \end{itemize}
 
 
 \end{itemize}

 
 }


\end{itemize}


\section{\underline{Decisiones de Diseño:} }


\begin{itemize}

\item  Al momento de trabajar en la clase Ruta se tenía que modelar que las rutas podían ser fijas o dinámicas y sus interacciones con los otros entes del UD. Para lograrlo surgieron dos opciones:

\begin{itemize}
\item[$o$] Crear la clase Ruta con el atributo tipo cuyo dominio sería “fija” y “dinámica”. Luego al tener la asociación “crea” entre Ruta y Usuario hubiese sido necesaria una restricción explícita, dado que un usuario solo puede crear rutas dinámicas. También, la asociación 'incluye' entre Ruta y Paquete hubiese necesitado una restricción dado que un paquete solo puede incluir una ruta fija.
\item[$o$] Crear la clase Ruta y especializarla con las clases Fija y Dinámica. Luego crear las asociaciones: 'crea' entre Usuario y Dinámica; 'incluye' entre Paquete y Usuario. Se decidió tomar la segunda opción ya que se considera un diseño más expresivo porque representa a las rutas fijas y dinámicas con la importancia que tienen , tomando en cuenta que las mismas tienen un comportamiento individual.

\end{itemize}
\item  Con respecto a la asociación valoración , se tenía que modelar la relación con Hito,Ciudad y Usuario. Al principio se modelaron las siguientes relaciones binarias: Valoración - Hito, Valoracion -Ciudad, Valoracion - Ruta y Usuario -Valoracion. Luego de un análisis, se decidió diseñar las siguientes relaciones ternarias : Usuario-Valoración - Hito , Usuario- Valoracion- Ciudad y Usuario- Valoración - Ruta. Con esa decision , es mas intuitivo comprender las relaciones.

\item La especialización de paquete trajo algunas decisiones importantes en el diseño. Un paquete relacionado con un usuario debe tomar en cuenta si el usuario desea visitar una ciudad con sus acompañantes. Estos acompañantes pueden ser niños, personas pertenecientes a la tercera edad o usuario estándar. Acompanante es un atributo de la asociación ‘contrata’ y esta diseñado para obtener el número y el tipo de acompañante que tiene un usuario, y de esta forma poder calcular un paquete con un precio especial para el usuario y sus acompañantes.
\item  Las rutas están conformadas por Hitos y Vías en un enfoque conceptual, y para modelar esto se comenzó creando una asociación entre Ruta e Hito y otra entre Ruta y Via. Luego se consideró que era más adecuado utilizar una agregación para ambos casos ya que hace que el modelo sea más autoexplicativo y expresivo dado que una ruta no está simplemente asociada a un hito o a una vía, sino que los mismo son parte de ella.


\end{itemize}


\section{\underline{Conclusión:} }





\hspace{0,4 cm} Una vez realizado el diagrama OMT (Object Modeling Technique), puede verse de forma clara y concisa cómo se comportan los entes en el universo de discurso. El diagrama representa un modelo de análisis y diseño de la base de datos YourCity. Además, se puede ver los atributos y las operaciones de cada clase y de cada asociación, los cuales juegan un papel fundamental en los requerimientos de la aplicación.

\hspace{0,4 cm} Durante la elaboración de este documento y el diseño del esquema conceptual OMT se presentaron una serie de dificultades al momento de realizar las restricciones explícitas y modelar de forma adecuada las clases y las asociaciones. Entre las dificultades principales que surgieron destaca la selección de como utilizar las herramientas del OMT para cumplir con las normas de calidad y simplificar el diagrama.


\hspace{0,4 cm}Como recomendaciones finales para futuras mejoras del diagrama OMT se recomienda mantener el portal lo más simple posible. Es importante tener en cuenta que lo más relevante en este trabajo es la realización de un diseño conceptual que cumpla con todos los requerimientos del universo de discurso y con las normas de calidad.





\vspace{10 cm}

\section{\underline{Bibliografía}}


\begin{itemize}
\item[$$] [1]  Soraya Abad-Mota, Lineamientos sobre cómo escribir informes técnicos, mayo 2007. 
\item[$$] [2] Soraya Abad-Mota, La Base de Datos Musical: Esquema Conceptual en el Modelo ER-E (Cuarta Edición). Abril 2013.
\item[$$][3] Soraya Abad-Mota, Paradigmas de Modelado de BD I (CI-5311): Caso de Estudio del Proyecto. Rutas Turísticas Urbanas (RTU), 24 de abril de 2013.
\item[$$][4] Michael Blaha and William Premerlani, Object­Oriented Modeling and Design for Database Applications. Prentice Hall, Upper Saddle River, New Yersey 07458. 
\end{itemize}


\end{document}